%%%%%%%%%%%%%%%%%%%%%%%%%%%%%%%%%%%%%%%%%%%%%%%%%%%%%%%%%%%%%%%%%%%%%%%%%%%%%%%%%%%%%%%%%%%%%%%%%%%%
% Data Types
%%%%%%%%%%%%%%%%%%%%%%%%%%%%%%%%%%%%%%%%%%%%%%%%%%%%%%%%%%%%%%%%%%%%%%%%%%%%%%%%%%%%%%%%%%%%%%%%%%%%

\section{Data Types}

We have different kinds of built-in \textit{things} in Python. These \textit{things} are called data
types.
\newline
Here is their table:

\begin{table}[H]
    \begin{tabular}{p{0.15\linewidth}p{0.25\linewidth}p{0.60\linewidth}}
        \toprule
        \textbf{Data Type} & \textbf{Category} & \textbf{Examples}\\
        \midrule

        \mintinline{python}{int} & Integers & 1, 2, 3, -1, -2, -3, 0\\
        \midrule

        \mintinline{python}{float} & Floating point numbers & 0.5, 1.5, 2.5, -0.5, -1.5, -2.5\\
        \midrule

        \mintinline{python}{bool} & Boolean & \mintinline{python}{False}, \mintinline{python}{True}\\
        \midrule

        \mintinline{python}{complex} & Complex numbers & \(1 + i\), \(2 + 2i\), \(-3 + 5i\)\\
        \midrule

        \mintinline{python}{str} & Text sequence type & \mintinline{python}{"a"},
                                                        \mintinline{python}{"abc"},
                                                        \mintinline{python}{"Hello, world!"}\\
        \midrule

        \mintinline{python}{list} & Sequence type & \mintinline{python}{[]},
                                                    \mintinline{python}{[0]},
                                                    \mintinline{python}{[1, 2, 3]},
                                                    \mintinline{python}{["ab", "bc", "cd"]}\\
        \midrule

        \mintinline{python}{tuple} & Sequence type & \mintinline{python}{()},
                                                     \mintinline{python}{(0)},
                                                     \mintinline{python}{(1, 2, 3)},
                                                     \mintinline{python}{("ab", "bc", "cd")}\\
        \midrule

        \mintinline{python}{range} & Sequence type & \mintinline{python}{range(10)},
                                                     \mintinline{python}{range(1, 10)},
                                                     \mintinline{python}{range(3, 8, 2)}}\\
        \midrule

        \mintinline{python}{set} & Set type & \mintinline{python}{{}},
                                              \mintinline{python}{{0}},
                                              \mintinline{python}{{1, 2, 3}},
                                              \mintinline{python}{{"ab", "bc", "cd"}}\\
        \midrule

        \mintinline{python}{frozenset} & Set type & \mintinline{python}{frozenset({})},
                                                    \mintinline{python}{frozenset({0, 1})}\\
        \midrule

        \mintinline{python}{dict} & Mapping type & \mintinline{python}{{}},
                                                   \mintinline{python}{{0: 1}},
                                                   \mintinline{python}{{"a": 2, "b": 3}}\\
        \bottomrule
    \end{tabular}
    \caption{Built-in Data Types}
    \label{tb.data.types}
\end{table}

\subsection{\mintinline{python}{int}}

Counting is something that we all do in our everyday lives. It would be rather inconvenient if
Python did not provide support for counting numbers (i.e., 1, 2, 3, etc.).
\\\\
On the other hand, counting numbers also have their negative counterparts and there's also zero.
Having them is equally important.
\\\\
Positive counting numbers, their counterparts, and zero, together make a set of numbers called
integers!
\\\\
Hence, we need to be able to represent integers. Python has \mintinline{python}{int} data type for
this purpose. Python's \mintinline{python}{int} data type can represent negative integers, zero, and
positive integers.

\subsection{\mintinline{python}{float}}

While we do have integers covered, we have not covered decimals. Python has our back!
\mintinline{python}{float} can be used to represent both positive and negative decimals such as -0.5
and 0.5.

\subsection{\mintinline{python}{bool}}

\mintinline{python}{bool} represents a boolean type. Boolean can only take two values -
\mintinline{python}{True} and \mintinline{python}{False}. That is all we need to know about this
type.
