%%%%%%%%%%%%%%%%%%%%%%%%%%%%%%%%%%%%%%%%%%%%%%%%%%%%%%%%%%%%%%%%%%%%%%%%%%%%%%%%%%%%%%%%%%%%%%%%%%%%
% The `print` Function
%%%%%%%%%%%%%%%%%%%%%%%%%%%%%%%%%%%%%%%%%%%%%%%%%%%%%%%%%%%%%%%%%%%%%%%%%%%%%%%%%%%%%%%%%%%%%%%%%%%%

\section{The \mintinline{python}{print} Function}

Just like in math, we have functions in Python as well. While we will cover Python functions in a
greater detail in later chapters, the \mintinline{python}{print} function is so useful that we will
start by learning how it works!

\begin{minted}{python}
    >>> print("Hello, world!")
    Hello, world!
\end{minted}

The way \mintinline{python}{print} works is that you write out these characters
\mintinline{python}{p-r-i-n-t}, followed by the left paren \mintinline{python}{(}, followed by
whatever we want to print, and finally the right paren \mintinline{python}{)}.

\begin{minted}{python}
    >>> print(0)
    0
    >>> print(1)
    1
    >>> print(9)
    9
    >>> print(10)
    10
\end{minted}
